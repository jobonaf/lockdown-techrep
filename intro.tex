Nei primi mesi del 2020 l'epidemia di COVID-19 ha coinvolto drammaticamente l'Italia. Per contenere il contagio, limitare l'impatto sul Sistema Sanitario Nazionale, ridurre i decessi, sono stati messi in atto provvedimenti di scala locale, regionale e nazionale. Tali provvedimenti hanno temporaneamente limitato la mobilità individuale e interrotto o ridotto alcune attività produttive e commerciali (Tab.\ref{tab:cronistoria}), determinando alcuni effetti sull'ambiente.

In questo rapporto analizziamo gli effetti ambientali registrati in Friuli Venezia Giulia relativi alla qualità dell'aria. Seguendo lo schema ``DPSIR''\footnote{schema Determinanti-Pressioni-Stato-Impatti-Risposte per la descrizione delle interazioni tra società e ambiente} approfondiamo 
\begin{itemize}
    \item \textbf{Determinanti}, cioè le variazioni nelle attività umane e dei parametri meteorologici più rilevanti dal punto di vista della qualità dell'aria;
    \item \textbf{Pressioni}, cioè le conseguenti variazioni nelle emissioni inquinanti in atmosfera;
    \item \textbf{Stato}, cioè come la composizione dell'aria che respiriamo è cambiata nelle settimane considerate.
\end{itemize}

Il periodo di studio copre febbraio e marzo 2020 per i determinanti antropici e le pressioni, estendendosi ad aprile per i determinanti meteorologici e gli indicatori di stato.

\begin{table}[ht]
    \centering
    \caption[Misure di contenimento del contagio e altri eventi rilevanti relativi a COVID-19]{Misure di contenimento del contagio e altri eventi rilevanti relativi a COVID-19 (aggiornamento: 4 giugno 2020)}
    \begin{tabular}{lll}
    \toprule
    \multicolumn{2}{l}{data} & descrizione \\
    \midrule
    \textbf{2019} &&\\
        dicembre    & 31 & le autorità cinesi riferiscono di un focolaio di polmoniti virali di origine ignota a Wuhan in Cina\\
 %   \midrule
    \textbf{2020} &&\\
        gennaio     & 30 & OMS dichiara l'emergenza sanitaria pubblica di interesse internazionale\\
                    & 31 & limitazioni ai voli da/per la Cina \\
                    &    & primi casi accertati COVID-19 in Italia\\
        febbraio    & 22 & ``zone rosse'' in alcuni Comuni (Lodigiano, Vo' Euganeo)\\
                    & 23 & chiusura delle scuole in alcune Regioni (incluso FVG)\\
                    & 25 & chiusura dei musei\\
        marzo       &  8 & \textit{lockdown} in Lombardia e in alcune aree di Veneto, Emilia-Romagna, Piemonte, Marche\\
                    &    & attenuazione delle limitazioni di mobilità e accesso nelle ``zone rosse''\\
                    &    & limitazione agli orari di apertura di alcuni esercizi \\
                    &  9 & \textit{lockdown} nazionale\\
                    &    & chiusura dei confini nazionali\\
                    &    & chiusura delle scuole in tutta Italia\\
                    & 11 & chiusura dei servizi pubblici\\
                    &    & OMS dichiara la pandemia\\
                    & 12 & chiusura di alcune categorie di esercizi (bar, ristoranti, ecc)\\
                    & 13 & resta attivo un solo aeroporto per ogni Regione\\
                    & 20 & chiusura dei parchi pubblici e dei cimiteri\\
                    & 23 & divieto di spostamento tra Comuni\\
                    &    & chiusura delle attività produttive non essenziali\\
        aprile      & 10 & riapertura di attività produttive che rispettino specifiche condizioni di sicurezza\\
                    & 11 & riapertura di alcuni esercizi (librerie, abbigliamento bambini, ecc)\\
        maggio      &  4 & maggiori possibilità di spostarsi tra Comuni e tra Regioni\\
                    &    & riapertura dei parchi pubblici\\
                    &    & riapertura di alcuni esercizi (\textit{take away})\\
                    &    & riapertura di alcune attività (manifatture, cantieri, ecc)\\
                    &    & incremento dei trasporti pubblici\\
                    & 18 & riapertura di alcuni esercizi (bar, ristoranti, ecc)\\
                    &    & possibilità di spostarsi all'interno della Regione\\
        giugno      &  3 & possibilità di spostarsi tra Regioni diverse\\
                    &    & riapertura dei confini nazionali agli altri Paesi UE\\
    \bottomrule                
    \end{tabular}
    \label{tab:cronistoria}
\end{table}

