L'esperienza drammatica e dolorosa del \textit{lockdown} nazionale, necessario per il contenimento della diffusione dell'epidemia COVID-19, ha determinato condizioni di studio ambientale uniche, che consentono di valutare dal vivo gli effetti sull'ambiente del blocco di interi settori della società.
Questa condizione eccezionale - e auspicabilmente irripetibile - ha stimolato ARPA-FVG ad avviare la raccolta e l'analisi di una serie articolata di dati e informazioni sulla variazione della mobilità, sui livelli di inquinamento atmosferico, sulla riduzione delle emissioni degli inquinanti e sulle condizioni meteo che hanno interessato il periodo di blocco delle attività. 

Dal punto di vista emissivo questo lavoro ha mostrato come le misure emergenziali a livello regionale abbiano complessivamente prodotto soprattutto \textbf{riduzioni nelle emissioni} degli ossidi di azoto (circa -25\%), seguiti dall’anidride carbonica (circa -19\%) e dal monossido di carbonio (circa -16\%). Decisamente minori sono risultate le riduzioni nelle emissioni di ammoniaca (circa -3\%) e di materiale particolato (circa -8\%). L’entità di queste riduzioni è coerente con il fatto che le misure adottate nella nostra regione hanno agito soprattutto sui trasporti. 

La riduzione delle emissioni inquinanti ha determinato un \textbf{calo delle concentrazioni in aria} di \textbf{biossido di azoto} (circa -40\% rispetto agli anni precedenti), registrato dalle stazioni di monitoraggio regionali, che sostanzialmente ha anticipato di tre o quattro settimane la consueta diminuzione delle concentrazioni che si osserva in primavera. Altrettanto marcata, nelle postazioni per il monitoraggio degli impatti del traffico, è risultata anche la riduzione nelle concentrazioni del \textbf{benzene}.
Le \textbf{polveri sottili} hanno presentato un calo decisamente meno rilevante (pari o inferiore al 10\%) e fluttuazioni più marcate, determinate dalla meteorologia e da un evento di trasporto di polveri desertiche tra il 27 e il 29 marzo. 
L'\textbf{ozono} - inquinante fortemente legato alla radiazione solare e dunque molto variabile tra un anno e l'altro - non ha mostrato variazioni evidenti rispetto agli anni precedenti, anche se l’andamento giornaliero sembra indicarne un leggero aumento nel periodo del \textit{lockdown}. 

Oltre all’andamento delle concentrazioni dei singoli inquinanti, i dati raccolti nel periodo del \textit{lockdown} sono stati utilizzati anche per valutare l’efficacia di alcuni \textbf{indicatori combinati}, ottenuti facendo il rapporto tra diversi inquinanti. Questi indicatori sono meno soggetti agli effetti confondenti della meteorologia.
Il rapporto toluene/benzene ad esempio in corrispondenza del \textit{lockdown} è calato vistosamente, da valori tipici del traffico veicolare a valori caratteristici della combustione di legna. Il rapporto tra gli ossidi totali di azoto e il monossido di azoto ha evidenziato che gli effetti più marcati si sono manifestati in prossimità delle strade. Infine l'analisi granulometrica delle polveri sottili ha palesato il venir meno, durante il \textit{lockdown}, della risospensione della frazione grossolana del particolato, indotta dal transito di veicoli e usualmente registrata dalle stazioni di bordo strada.

L'analisi del contenuto di alcuni \textbf{metalli} nelle polveri sottili evidenzia un significativo calo di antimonio e rame, originati prevalentemente dall'usura dei freni dei veicoli. Viene inoltre confermata l'origine dal deserto vicino al Mar Caspio delle notevoli masse di polveri che hanno interessato tutta la regione tra il 27 e il 29 marzo.
Anche sul contenuto di \textbf{idrocarburi policiclici aromatici} (IPA) nel PM10 le azioni di contenimento del contagio hanno determinato effetti misurabili. Agendo in particolare sul traffico, hanno determinato la riduzione degli IPA associati ai trasporti, ma facendo risaltare quelli principalmente legati alla combustione domestica.

In conclusione dunque la riduzione delle emissioni dei trasporti porta evidenti benefici ambientali, ma per alcuni inquinanti non è sufficiente, neppure se applicata su grande scala e per lungo tempo. Per il miglioramento della qualità dell'aria sono necessarie perciò strategie che agiscano anche su altri settori, quali il riscaldamento domestico, l'agricoltura e l'industria.

\vfill
Questo lavoro è stato possibile anche grazie al progetto europeo LIFE-PREPAIR.