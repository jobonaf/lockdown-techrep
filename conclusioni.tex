I provvedimenti di confinamento e di limitazione della mobilità, messi in atto a livello locale, regionale e nazionale (tab.\ref{tab:cronistoria}) per contenere il contagio di COVID-19, hanno determinato alcuni effetti sulla matrice ambientale aria.

I flussi di traffico sono diminuiti progressivamente (fig.\ref{fig:riduzionedeterminanti}) a partire dalla quarta settimana di febbraio, fino a raggiungere nelle ultime settimane di marzo riduzioni - rispetto alle condizioni normali - nelle aree urbane di oltre il 70\%, sulle strade extraurbane e autostrade di oltre l'80\% dei veicoli leggeri e del 50-70\% dei veicoli pesanti.
Nel frattempo, anche il traffico aereo è stato drasticamente ridotto  (fig.\ref{fig:riduzionedeterminanti}), fino ad azzerarsi nelle ultime settimane di marzo. Le variazioni delle attività industriali e portuali e dei consumi energetici per il riscaldamento domestico saranno oggetto di ulteriori approfondimenti.

Riduzioni così marcate di alcune attività antropiche hanno determinato flessioni significative nelle emissioni di alcuni inquinanti  (fig.\ref{fig:riduzioneemissioni}), stimate tenendo conto delle riduzioni di traffico stradale e aereo nell'ordine del 25\% per gli ossidi di azoto, del 19\% per l'anidride carbonica, del 16\% per il monossido di carbonio. Decisamente minori sono risultate le riduzioni delle emissioni dei composti organici volatili (circa -4\%), dell'ammoniaca (-3\%) e di polveri sottili (circa -8\%). L'entità di queste riduzioni è coerente con il fatto che le misure adottate nella nostra regione hanno agito soprattutto sui trasporti. Il probabile aumento dei consumi per il riscaldamento domestico potrebbe però aver parzialmente compensato la riduzione di emissioni di polveri sottili.

La riduzione delle emissioni inquinanti ha determinato un calo delle concentrazioni di biossido di azoto (circa -40\% rispetto agli anni precedenti), registrato dalle stazioni di monitoraggio regionali, che sostanzialmente ha anticipato di tre o quattro settimane la consueta diminuzione delle concentrazioni che si osserva in primavera  (figg.\ref{fig:andaminq} e \ref{fig:giornino2}). Tale riduzione è ascrivibile alle azioni di contenimento, poiché si manifesta attraverso una brusca attenuazione dei picchi corrispondenti alle ore di punta del traffico  (fig.\ref{fig:noxgiorni}), ma ad essa hanno anche contribuito le condizioni di vivace ventilazione, specie nell'area triestina  (figg.\ref{fig:RSV} e \ref{fig:giornivento}). Altrettanto rilevante, nelle postazioni per il monitoraggio degli impatti del traffico, è stato il calo delle concentrazioni di benzene, che si aggiunge alla diminuzione già osservata negli ultimi anni per questo inquinante  (figg.\ref{fig:andaminq} e \ref{fig:giornibenzene}).

Le polveri sottili hanno presentato un calo decisamente meno rilevante (pari o inferiore al 10\%) e fluttuazioni più marcate, determinate dalla meteorologia e da un evento di trasporto di polveri desertiche tra il 27 e il 29 marzo (figg.\ref{fig:andaminq} e \ref{fig:giornipm10}). 

L'ozono - inquinante fortemente legato alla radiazione solare e dunque molto variabile tra un anno e l'altro - non ha mostrato variazioni rilevanti rispetto agli anni precedenti, ma l'aumento delle concentrazioni in aprile potrebbe essere stato favorito in parte dalle azioni di contenimento  (figg.\ref{fig:andaminq} e \ref{fig:giornio3}). 

L'analisi condotta sui rapporti tra le concentrazioni di inquinanti ha consentito di ridurre l'effetto confondente della meteorologia e ha messo così in luce alcuni fenomeni interessanti. Il rapporto toluene/benzene - buon indicatore dell'origine prevalente delle emissioni locali di questi composti organici volatili - nelle settimane della serrata è calato vistosamente, seppur con ampie fluttuazioni, da valori tipici del traffico veicolare a valori caratteristici della combustione di legna (pag.\pageref{cap:tb} e sgg.). A fine aprile - in corrispondenza del mitigarsi delle temperature e dell'allentarsi delle misure di contenimento - tale indicatore è tornato a livelli più consueti. L'analisi del rapporto NOx/NO - il cui valore dipende sia dalla distanza delle fonti emissive sia dalla cinetica chimica - mostra che per gli ossidi di azoto gli effetti del \textit{lockdown} si sono manifestati soprattutto in prossimità delle strade  (pag.\pageref{cap:noxno} e sgg.). Analogamente, l'analisi granulometrica delle polveri sottili evidenzia il venir meno durante la serrata della risospensione di polveri grossolane, indotta dal transito di veicoli e usualmente registrata dalle stazioni di bordo strada (pag.\pageref{cap:cef} e sgg.).

L'analisi del contenuto di alcuni metalli nelle polveri sottili campionate a Udine, presenti in basse concentrazioni già prima dell'entrata in vigore delle azioni di contenimento, evidenzia un significativo calo di antimonio e rame, originati prevalentemente dall'usura dei freni dei veicoli, e conferma l'origine terrigena (deserti asiatici) delle notevoli masse di PM10 che hanno interessato tutta la regione tra il 27 e il 29 marzo (pag.\pageref{cap:metalli} e sgg.).

Anche sul contenuto di idrocarburi policiclici aromatici (IPA) nel PM10 le azioni di contenimento del contagio hanno determinato effetti misurabili. Agendo in particolare sul traffico, hanno determinato la riduzione degli IPA associati ai trasporti, facendo così risaltare quelli principalmente legati alla combustione domestica (pag.\pageref{cap:ipa} e sgg.).

\paragraph{Prospettive}
L'analisi degli effetti del \textit{lockdown} sulla matrice aria richiederà ulteriori approfondimenti ed estensioni temporali nei prossimi mesi, quando saranno disponibili più dati e sarà possibile valutare anche gli effetti della ripresa di molte attività.

La necessità di individuare i nessi causali tra riduzioni emissive ed effetti sulle concentrazioni in aria ha richiesto l'acquisizione tempestiva di informazioni sulle attività antropiche di interesse, l'elaborazione di specifici indicatori e l'uso di strumenti di analisi multi-variata. Tali flussi di dati e metodiche di analisi hanno rivelato un notevole potenziale a supporto dell'interpretazione dei fenomeni in atto. Pertanto si auspica lo sviluppo di collaborazioni e competenze in tal senso, che in futuro potrebbero rivelarsi fondamentali per valutare gli effetti di azioni strutturali e pianificate per il miglioramento della qualità dell'aria.

Lo studio ha evidenziato che la riduzione delle emissioni dei trasporti porta evidenti benefici ambientali, ma per alcuni inquinanti non è sufficiente, neppure se applicata su grande scala e per lungo tempo. Per il miglioramento della qualità dell'aria è necessario continuare ad intervenire anche su altri settori, quali il riscaldamento domestico, l'agricoltura e l'industria.  

